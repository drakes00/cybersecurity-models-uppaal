\documentclass[10pt,a4paper]{article}
\usepackage[utf8]{inputenc}
\usepackage[francais]{babel}
\usepackage[T1]{fontenc}
\usepackage{amsmath}
\usepackage{amsfonts}
\usepackage{amssymb}
\usepackage{graphicx}
\usepackage{caption}
\usepackage{geometry}
\usepackage{hyperref}
\usepackage[underline=true,rounded corners=false]{pgf-umlsd}


\title{Scénarios d'attaques}
\author{}
\date{}
\begin{document}
\maketitle
\section{Configuration 1:}
Un seul serveur MODBUS ou OPC-UA (configuré en mode None) contrôle l’ensemble du procédé. Il communique avec un client MODBUS ou OPC-UA qui contrôle l’arrêt et le démarrage du procédé et lit l’état des capteurs, du tapis roulant et de la valve.
\medskip
\subsection{Résultat :}

Dans cette configuration toutes les attaques sont possibles:
\medskip

\textit{Attaque 1 :} Forcer l’ouverture de la valve quel que soit l’état des capteurs. Si une
bouteille était présente, le liquide finira par déborder, sinon il coulera hors de la bouteille.
\medskip
%%%%%%%%%%%%% Scénario 1%%%%%%%%%%%%%%%%%%%%%%%%%%

\textbf{Scénario 1 :}
\medskip
\medskip

\begin{sequencediagram}
  \tikzstyle{inststyle}+=[{font=\large}]
  \def\unitfactor{.9}

  \newinst{instance 1}
  {Client}

  \newinst[4cm]{instance 2}
  {Attaquant}

  \newinst[4cm]{instance 3}
  {Serveur}

  \tikzstyle{instcolordienst}=[fill=black!25]
  \tikzstyle{instcolorbuerger}=[fill=black!25]

  \messcall{instance 1}{écrire(run:=true)}{instance 3}
  \messcall{instance 3}{OK}{instance 1}
  \messcall{instance 1}{lire(motor)}{instance 2}
  \begin{callself}
      {instance 2}{Modifier(lire(motor))}{renvoi écrire(nozzle:=true)}
    \end{callself}
  \messcall{instance 2}{écrire(nozzle:=true)}{instance 3}
  
\end{sequencediagram}
\newpage 

%%%%%%%%%%%%% Scénario 2%%%%%%%%%%%%%%%%%%%%%%%%%%

\textbf{Scénario 2 :}
\medskip
\medskip

\begin{sequencediagram}
  \tikzstyle{inststyle}+=[{font=\large}]
  \def\unitfactor{.9}

  \newinst{instance 1}
  {Client}

  \newinst[4cm]{instance 3}
  {Serveur}
  
  \newinst[4cm]{instance 2}
  {Attaquant}


  \tikzstyle{instcolordienst}=[fill=black!25]
  \tikzstyle{instcolorbuerger}=[fill=black!25]

  \messcall{instance 1}{écrire(run:=true)}{instance 3}
  \messcall{instance 3}{OK}{instance 1}
  \begin{callself}
      {instance 2}{Créer()}{renvoi nozzle:=true}
    \end{callself}
  \messcall{instance 2}{écrire(nozzle:=true)}{instance 3}
  
\end{sequencediagram}

\textit{Attaque 2 :} Forcer le tapis roulant à avancer quel que soit l’état des capteurs. Les
bouteilles avanceront vides.
\medskip

%%%%%%%%%%%%% Scénario 1%%%%%%%%%%%%%%%%%%%%%%%%%%

\textbf{Scénario 1 :}
\medskip
\medskip

\begin{sequencediagram}
  \tikzstyle{inststyle}+=[{font=\large}]
  \def\unitfactor{.9}

  \newinst{instance 1}
  {Client}

  \newinst[4cm]{instance 2}
  {Attaquant}

  \newinst[4cm]{instance 3}
  {Serveur}

  \tikzstyle{instcolordienst}=[fill=black!25]
  \tikzstyle{instcolorbuerger}=[fill=black!25]

  \messcall{instance 1}{écrire(run:=true)}{instance 3}
  \messcall{instance 3}{OK}{instance 1}
  \messcall{instance 1}{lire(motor)}{instance 2}
  \begin{callself}
      {instance 2}{Modifier(lire(motor))}{renvoi écrire(motor:=true)}
    \end{callself}
  \messcall{instance 2}{écrire(motor:=true)}{instance 3}
  
\end{sequencediagram}
\newpage 

%%%%%%%%%%%%% Scénario 2%%%%%%%%%%%%%%%%%%%%%%%%%%

\textbf{Scénario 2 :}
\medskip
\medskip

\begin{sequencediagram}
  \tikzstyle{inststyle}+=[{font=\large}]
  \def\unitfactor{.9}

  \newinst{instance 1}
  {Client}

  \newinst[4cm]{instance 3}
  {Serveur}
  
  \newinst[4cm]{instance 2}
  {Attaquant}


  \tikzstyle{instcolordienst}=[fill=black!25]
  \tikzstyle{instcolorbuerger}=[fill=black!25]

  \messcall{instance 1}{écrire(run:=true)}{instance 3}
  \messcall{instance 3}{OK}{instance 1}
  \begin{callself}
      {instance 2}{Créer()}{renvoi motor:=true}
    \end{callself}
  \messcall{instance 2}{écrire(motor:=true)}{instance 3}
  
\end{sequencediagram}
\textit{Attaque 3 :} Forcer l’ouverture de la valve et le tapis roulant à avancer quel que soit l’état des capteurs. Cette attaque est un mélange des attaques 1 et 2.
\medskip

\textbf{Scénarios:} les mêmes scénarios de l'attaque 1 et 2.

\section{Configuration 2 :}
\paragraph{}
Un seul serveur OPC-UA (configuré en mode Sign ou SignEncrypt) contrôle l’ensemble du procédé et communique avec un client OPC-UA qui contrôle l’arrêt et le démarrage du procédé et lit l’état des capteurs, du tapis roulant et de la valve.
\subsection{Résultat :}
je n'ai pas trouvé des attaques.
\section{Configuration 3 :}
\paragraph{}
Le tapis roulant et son capteur sont contrôlés par un serveur MODBUS ou OPC-UA (configuré en mode None) et la valve et sont capteur par un serveur OPC-UA (configuré en mode Sign ou SignEncrypt). Un unique client multi-protocoles communique avec les deux serveurs.
\subsection{Résultat :}
Dans cette configuration toutes les attaques sont possibles:
\medskip

\textit{Attaque 1 :} Forcer l’ouverture de la valve quel que soit l’état des capteurs. Si une
bouteille était présente, le liquide finira par déborder, sinon il coulera hors de la bouteille.
\medskip
\begin{sequencediagram}
  \tikzstyle{inststyle}+=[{font=\large}]
  \def\unitfactor{.9}

  \newinst{instance 1}
  {Client}

  \newinst[2cm]{instance 2}
  {Attaquant}

  \newinst[2cm]{instance 3}
  {OPC-UA}
  
  \newinst[2cm]{instance 4}
  {MODBUS}

  \tikzstyle{instcolordienst}=[fill=black!25]
  \tikzstyle{instcolorbuerger}=[fill=black!25]

  \messcall{instance 1}{écrire(run:=1)}{instance 3}
  \messcall{instance 3}{OK}{instance 1}
  \messcall{instance 1}{lire(motor)}{instance 2}
  \begin{callself}
      {instance 2}{Modifier(lire(motor))}{renvoi écrire(bottle:=true)}
    \end{callself}
  \messcall{instance 2}{écrire(bottle:=true)}{instance 4}
\end{sequencediagram}
\medskip

\textit{Attaque 2 :} Forcer le tapis roulant à avancer quel que soit l’état des capteurs. Les
bouteilles avanceront vides.
\medskip

\begin{sequencediagram}
  \tikzstyle{inststyle}+=[{font=\large}]
  \def\unitfactor{.9}

  \newinst{instance 1}
  {Client}

  \newinst[2cm]{instance 2}
  {Attaquant}

  \newinst[2cm]{instance 3}
  {OPC-UA}
  
  \newinst[2cm]{instance 4}
  {MODBUS}

  \tikzstyle{instcolordienst}=[fill=black!25]
  \tikzstyle{instcolorbuerger}=[fill=black!25]

  \messcall{instance 1}{écrire(run:=1)}{instance 3}
  \messcall{instance 3}{OK}{instance 1}
  \messcall{instance 1}{lire(motor)}{instance 2}
  \begin{callself}
      {instance 2}{Modifier(lire(motor))}{renvoi écrire(motor:=true)}
    \end{callself}
  \messcall{instance 2}{écrire(motor:=true)}{instance 4}
\end{sequencediagram}
\medskip
\newpage
\textit{Attaque 3 :} Forcer l’ouverture de la valve et le tapis roulant à avancer quel que soit l’état des capteurs. Cette attaque est un mélange des attaques 1 et 2.
\medskip

\begin{sequencediagram}
  \tikzstyle{inststyle}+=[{font=\large}]
  \def\unitfactor{.9}

  \newinst{instance 1}
  {Client}

  \newinst[2cm]{instance 2}
  {Attaquant}

  \newinst[2cm]{instance 3}
  {OPC-UA}
  
  \newinst[2cm]{instance 4}
  {MODBUS}

  \tikzstyle{instcolordienst}=[fill=black!25]
  \tikzstyle{instcolorbuerger}=[fill=black!25]

  \messcall{instance 1}{écrire(run:=1)}{instance 3}
  \messcall{instance 3}{OK}{instance 1}
  \messcall{instance 1}{lire(motor)}{instance 2}
  \begin{callself}
      {instance 2}{Modifier(lire(motor))}{renvoi écrire(motor:=true)}
    \end{callself}
  \messcall{instance 2}{écrire(motor:=true)}{instance 4}
\end{sequencediagram}
\end{document}