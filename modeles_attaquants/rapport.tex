\documentclass[11pt,a4paper,twoside,openright]{scrreprt}
\usepackage[utf8]{inputenc}
\usepackage{ae}
\usepackage{aecompl}
\usepackage[T1]{fontenc}
\setcounter{tocdepth}{3}
\usepackage{graphicx}
\author{Abdelaziz KHALED}
\title{Les modèles d'attaquants}
\renewcommand{\thesection}{\arabic{section}}

\begin{document}
\let\cleardoublepage\clearpage
\maketitle
\tableofcontents
\newpage

\section{Introduction}

\section{Modèles d'attaquants}
\subsection{Clauses de Horn}
\paragraph{}
Modélisation et limites

\subsection{Le Modèle Dolev-Yoa}
Les caractéristiques de modèle Dolev-Yao\\
Les extensions de modèle Dolev-Yao	
	
\subsection{Formalisation de différents modèles d'attaquants}
\section{Modèles d'attaquants avec UPPAAL}
\subsection{Le modèle UPPAAL}
un aperçu sur UPPAAL
\subsection{La modélisation de différents modèles d'attaquants}
Lien entre les clauses de Horn et les automates(Transformation)\\
Explication la modélisation de chaque attaquant.
\section{Comparaison entre CSP,Pi-calcul et clauses de Horn}
les limites de clauses de Horn par rapport à CSP et Pi-calcul
\section{Conclusion}
\bibliography{mabiblio}
\bibliographystyle{plain}
\end{document}