\documentclass[10pt,a4paper]{article}
\usepackage[utf8]{inputenc}
\usepackage[francais]{babel}
\usepackage[T1]{fontenc}
\usepackage{amsmath}
\usepackage{amsfonts}
\usepackage{amssymb}
\usepackage{graphicx}
%%%%%%%% Package%%%%%%%%%%%%%%
\usepackage{xcolor}
\usepackage{graphicx}
\usepackage{tikz}
\usepackage{multicol}
\usepackage{listings}
\usetikzlibrary{arrows,positioning}

\author{Abdelaziz KHALED}
\begin{document}
\section{Model Checking}
\paragraph{}
Model checking is a technique for automatically verifying a finite state concurrent systems. It was successfully used in practice to verify embedded systems and communication protocols\cite{ref1}.The model checking requires the model of the system $M$,the specification of property $\Phi$ and an algorithm who checks automatically if the model satisfy the
property or not. 
\begin{center}
\begin{figure}[h]

\centering
\begin{tikzpicture}
[node distance = 1cm, auto,font=\footnotesize,
% STYLES
every node/.style={node distance=3cm},
% The comment style is used to describe the characteristics of each force
comment/.style={rectangle, inner sep= 5pt, text width=2cm, node distance=2cm, font=\scriptsize\sffamily},
% The force style is used to draw the forces' name
force/.style={rectangle, draw, fill=black!10, inner sep=2pt, text width=2cm, text badly centered, minimum height=1.2cm,node distance=1.5cm}] 

% Draw forces
\node [force] (generator) {Model Checking};
\node [force, right=3cm of generator ,above of=generator] (property) {Security Property};
\node [force, left=4cm of property] (model) {System Model};
%\node [comment, below of=generator] (ag) {};
\node [force,right=3cm of generator ,below of=generator] (no) {No + counter example};
\node [force,left=4cm of no] (yes) {Yes};
% Draw the links between forces
\path[->,thick] 
(property) edge (generator)
(model) edge (generator)
%(generator) edge (ag)
(generator) edge (no)
(generator) edge (yes);

\end{tikzpicture} 
\caption{Schematic view of the model-checking approach.}
\end{figure}
\end{center}

\section{The UPPAAL model and tool:}
\paragraph{}
Uppaal is a tool box for validation (via graphical simulation) and verification (via automatic model-checking) of real-time systems\cite{ref2}.This project has been developed in partnership with Uppsala university and Aalborg university. It composed of two main parts a graphical user interface (GUI) implemented by Java and a model-checker engine implemented by C++.
\paragraph{}
The graphical interface (GUI) is subdivided to three parts: the editor, the simulator and the verifier.the editor is divided into two parts one gives access to different templates and declarations, while the other is using to draw the model.The simulator its role is the execution of the system either manual or automatic.the verifier its role is checking the properties.
\paragraph{}
The model checker UPPAAL is based on timed automata for system modeling and temporal logic CTL for the specification of the properties, also we can declare variables,array,clocks and channels.
\paragraph{}
In UPPAAL we find three types of locations : the first is the normal location, the second is the committed location, and the last is the urgent location, to relate  between locations we use the edges, and every edge contains four variables : select,guard,synchronize, and update. Also for communication between processes there are three channels: binary synchronization channels, broadcast channels and urgent synchronization channels.
\paragraph{}
UPPAAL uses a simplified version of CTL. The syntax is the following :
\[Prop::=A[]p | E<>p | E[]p | A<>p | p->q\]
\paragraph{}					
A[]P expresses p should true in all reachable states. E<>P sometimes p true. E[]p there is a path where p always is true. A<>P In each path there is a state where p is true. p->q, p imply q. 

\bibliography{mabiblio}
\bibliographystyle{plain}

\end{document}