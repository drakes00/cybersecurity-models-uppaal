\documentclass[10pt,a4paper]{article}
\usepackage[utf8]{inputenc}
\usepackage[francais]{babel}
\usepackage[T1]{fontenc}
\usepackage{amsmath}
\usepackage{amsfonts}
\usepackage{amssymb}
\usepackage{multirow}
\usepackage{rotating}
\usepackage{tikz}
\usepackage{pifont}
\newcommand{\cmark}{\ding{51}}%
\newcommand{\xmark}{\ding{55}}%
\author{}
\date{}
\title{Les résultats de l’expérimentation}
\begin{document}
\maketitle
Ce tableau représente les résultats de l’expérimentation avec le temps maximal pour trouver l’attaque :\newline

%\begin{table}[]
\small
\begin{tabular}{ll|c|c|c|c|c|}
\cline{3-7}
&  &  Attaquant 1 &  Attaquant 2 &  Attaquant 3 & Attaquant 4 & Temps  \\ \hline
\multicolumn{1}{|l|}{\multirow{3}{*}{Conf1}} & Attaque1 &  \color{green}{\cmark} & \color{green}{\cmark} & \color{green}{\cmark} & \color{red}{\xmark}  & 0.99 S  \\ \cline{2-7} 
\multicolumn{1}{|l|}{}& Attaque2 & \color{green}{\cmark} & \color{green}{\cmark} & \color{green}{\cmark} & \color{red}{\xmark} & 1.21 S  \\ \cline{2-7} 
\multicolumn{1}{|l|}{}& Attaque3 & \color{green}{\cmark} & \color{green}{\cmark} & \color{green}{\cmark} & \color{red}{\xmark} & 0.13 S  \\ \hline
\multicolumn{1}{|l|}{\multirow{3}{*}{Conf3}} & Attaque1 & \color{green}{\cmark} & \color{green}{\cmark} & \color{green}{\cmark} & \color{red}{\xmark} &  11.99 S\\ \cline{2-7} 
\multicolumn{1}{|l|}{}& Attaque2 & \color{green}{\cmark} & \color{green}{\cmark} & \color{green}{\cmark} & \color{red}{\xmark} & 19.14 S  \\ \cline{2-7} 
\multicolumn{1}{|l|}{}& Attaque3 & \color{green}{\cmark} & \color{green}{\cmark} & \color{green}{\cmark} & \color{red}{\xmark} & 18.34  S \\ \hline
\end{tabular}
%\end{table}
\medskip

Dans la deuxième expérimentation j’ai changé les capacités de l’attaquant 3. Il peut forger les messages seulement pour contrôler la valve ou le capture.\newline 

%\begin{table}[]
\small
\begin{tabular}{ll|c|c|c|c|c|}
\cline{3-7}
&  &  Attaquant 1 &  Attaquant 2 &  Attaquant 3 & Attaquant 4 & Temps  \\ \hline
\multicolumn{1}{|l|}{\multirow{3}{*}{Conf1}} & Attaque1 &  \color{green}{\cmark} & \color{green}{\cmark} & \color{green}{\cmark} & \color{red}{\xmark}  & 0.99 S  \\ \cline{2-7} 
\multicolumn{1}{|l|}{}& Attaque2 & \color{green}{\cmark} & \color{green}{\cmark} & \color{green}{\cmark} & \color{red}{\xmark} & 1.21 S  \\ \cline{2-7} 
\multicolumn{1}{|l|}{}& Attaque3 & \color{green}{\cmark} & \color{green}{\cmark} & \color{green}{\cmark} & \color{red}{\xmark} & 0.13 S  \\ \hline
\multicolumn{1}{|l|}{\multirow{3}{*}{Conf3}} & Attaque1 & \color{green}{\cmark} & \color{green}{\cmark} & \color{red}{\xmark} & \color{red}{\xmark} &  11.99 S\\ \cline{2-7} 
\multicolumn{1}{|l|}{}& Attaque2 & \color{green}{\cmark} & \color{green}{\cmark} & \color{red}{\xmark} & \color{red}{\xmark} & 19.14 S  \\ \cline{2-7} 
\multicolumn{1}{|l|}{}& Attaque3 & \color{green}{\cmark} & \color{green}{\cmark} & \color{red}{\xmark} & \color{red}{\xmark} & 18.34  S \\ \hline
\end{tabular}
%\end{table}
\medskip

Dans la troisième expérimentation j’ai modifié le client et l’attaquant 2. Pour le client il peut lire seulement l’état des capteurs et pour l’attaquant peut modifier les champs : fonction(lire, écrire) et valeur(vrai et faux).\newline 

%\begin{table}[]
\small
\begin{tabular}{ll|c|c|c|c|c|}
\cline{3-7}
&  &  Attaquant 1 &  Attaquant 2 &  Attaquant 3 & Attaquant 4 & Temps  \\ \hline
\multicolumn{1}{|l|}{\multirow{3}{*}{Conf1}} & Attaque1 &  \color{green}{\cmark} & \color{green}{\cmark} & \color{green}{\cmark} & \color{red}{\xmark}  & 0.99 S  \\ \cline{2-7} 
\multicolumn{1}{|l|}{}& Attaque2 & \color{green}{\cmark} & \color{green}{\cmark} & \color{green}{\cmark} & \color{red}{\xmark} & 1.21 S  \\ \cline{2-7} 
\multicolumn{1}{|l|}{}& Attaque3 & \color{green}{\cmark} & \color{red}{\xmark} & \color{green}{\cmark} & \color{red}{\xmark} & 0.13 S  \\ \hline
\multicolumn{1}{|l|}{\multirow{3}{*}{Conf3}} & Attaque1 & \color{green}{\cmark} & \color{green}{\cmark} & \color{red}{\xmark} & \color{red}{\xmark} &  11.99 S\\ \cline{2-7} 
\multicolumn{1}{|l|}{}& Attaque2 & \color{green}{\cmark} & \color{red}{\xmark} & \color{red}{\xmark} & \color{red}{\xmark} & 19.14 S  \\ \cline{2-7} 
\multicolumn{1}{|l|}{}& Attaque3 & \color{green}{\cmark} & \color{red}{\xmark} & \color{red}{\xmark} & \color{red}{\xmark} & 18.34  S \\ \hline
\end{tabular}
%\end{table}
\medskip

\textbf{La spécification des propriétés :}
\begin{equation}
AG\; not(nozzle==true\; and\; bottle==false)
\end{equation}
\begin{equation}
AG\; not(motor==true\; and\; bottle==true\; and\; level==false)
\end{equation}
\begin{equation}
AG\; not(nozzle==true \; and\; motor==true)
\end{equation}
\medskip

pour trouver la première attaque dans la configuration 3 j'ai utilisé cette requête : 
\[AG\; not(nozzle==true\; and\; action==true)\]
La variable $action$ prend la valeur true si le serveur change l'état de capture $bottle$ ou l'état de moteur.

\end{document}