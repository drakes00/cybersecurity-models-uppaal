\documentclass[10pt,a4paper]{article}
\usepackage[utf8]{inputenc}
\usepackage[francais]{babel}
\usepackage[T1]{fontenc}
\usepackage{amsmath}
\usepackage{amsfonts}
\usepackage{amssymb}
\usepackage{multirow}
\usepackage{rotating}
\usepackage{tikz}
\usepackage{pifont}
\newcommand{\cmark}{\ding{51}}%
\newcommand{\xmark}{\ding{55}}%
\author{}
\date{}
\title{Les résultats de l’expérimentation}
\begin{document}
\maketitle
\medskip
\textbf{Modifications :}
\begin{itemize}
\item Attaquant 2 peut modifier les champs : variable,valeur.
\item Attaquant 3 peut forger les messages seulement pour contrôler le moteur.\newline 
\end{itemize}

\textbf{Propriétés à vérifier :}
\begin{enumerate}
\item La valve ne s'ouvre que lorsqu'une bouteille est en position.
\item Le moteur ne se démarre que lorsque la bouteille est pleine.
\item La valve ne s'ouvre que lorsque le moteur s'arr\^{e}te.
\item La valve ne s'ouvre pas lorsque le serveur force le changement de l'état du capteur de position.
\end{enumerate}
\medskip

\textbf{La spécification des propriétés :}
\begin{equation}
AG\; not(nozzle==true\; and\; bottle==false)
\end{equation}
\begin{equation}
AG\; not(motor==true\; and\; bottle==true\; and\; level==false)
\end{equation}
\begin{equation}
AG\; not(nozzle==true \; and\; motor==true)
\end{equation}
\begin{equation}
AG\; not(nozzle==true \; and\; action==true)
\end{equation}

\medskip
%\begin{table}[]
\small
\begin{tabular}{ll|c|c|c|c|}
\cline{3-6}
&  &  Attaquant 1 &  Attaquant 2 &  Attaquant 3 & Attaquant 4 \\ \hline
\multicolumn{1}{|l|}{\multirow{4}{*}{Conf 1}} & Propriété 1 &  \color{green}{\cmark} & \color{green}{\cmark} & \color{red}{\xmark} & \color{red}{\xmark}    \\ \cline{2-6} 
\multicolumn{1}{|l|}{}& Propriété 2 & \color{green}{\cmark} & \color{green}{\cmark} & \color{green}{\cmark} & \color{red}{\xmark}   \\ \cline{2-6} 
\multicolumn{1}{|l|}{}& Propriété 3 & \color{green}{\cmark} & \color{green}{\cmark} & \color{green}{\cmark} & \color{red}{\xmark}   \\\cline{2-6}
\multicolumn{1}{|l|}{}& Propriété 4 & \color{green}{\cmark} & \color{green}{\cmark} & \color{red}{\xmark} & \color{red}{\xmark}   \\ \hline

\multicolumn{1}{|l|}{\multirow{4}{*}{Conf 3}} & Propriété 1 & \color{red}{\xmark} & \color{red}{\xmark} & \color{red}{\xmark} & \color{red}{\xmark} \\ \cline{2-6} 
\multicolumn{1}{|l|}{}& Propriété 2 & \color{green}{\cmark} & \color{green}{\cmark} & \color{green}{\cmark} & \color{red}{\xmark} \\ \cline{2-6} 
\multicolumn{1}{|l|}{}& Propriété 3 & \color{green}{\cmark} & \color{green}{\cmark} & \color{green}{\cmark} & \color{red}{\xmark} \\ \cline{2-6}
\multicolumn{1}{|l|}{}& Propriété 4 & \color{green}{\cmark} & \color{green}{\cmark} & \color{red}{\xmark} & \color{red}{\xmark} \\ \hline
\end{tabular}
%\end{table}
\paragraph{}
Dans la configuration 3 la propriété 1 n'a pas violé parceque le serveur OPC-UA qui contr\^{o}le la valve et l'attaquant n'a pas accès à la valve.  

\end{document}